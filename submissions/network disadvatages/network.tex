\documentclass{article}
\usepackage{geometry}
\geometry{letterpaper, top=1in, bottom=1in, left=1in, right=1in}
\title{Network Disadvantages}
\date{\today}
\author{%
    \begin{tabular}{@{}lll@{}}
        Abdallah Tantawy & BH23500021 & BSCS \\
        Mohammed Ali & BH23500016 & BSCS \\
        Khaled Abdulla & BH23500207 & BSCS \\
    \end{tabular}%
}

\begin{document}
\maketitle


\section*{Disadvantages of Different Types of Networks}

\subsection*{Personal Area Network (PAN)}
A Personal Area Network (PAN) typically covers a very small area, such as a room or a personal workspace. One significant disadvantage of PANs is their limited range, which can be a disadvantage if you need to connect devices that are not in close proximity. PANs are generally not suitable for connecting devices over long distances, and they may require additional equipment or infrastructure to extend their range.

\subsection*{Local Area Network (LAN)}
Local Area Networks (LANs) are limited in terms of their geographical coverage. A disadvantage of LANs is that they are typically confined to a single building or a limited geographic area. This can be a limitation when organizations need to connect multiple locations that are spread out over a wider area. In such cases, LANs would require additional networking technologies or be supplemented by other types of networks, which can be complex and costly.

\subsection*{Metropolitan Area Network (MAN)}
Metropolitan Area Networks (MANs) are designed to cover a larger geographic area than LANs but are still limited in range. A disadvantage of MANs is that they may not be suitable for connecting locations that are spread across an entire city or metropolitan area. Setting up and maintaining a MAN can be expensive, and it may not provide the coverage needed for extensive citywide connectivity.

\subsection*{Wide Area Network (WAN)}
While Wide Area Networks (WANs) offer extensive coverage and connectivity over vast geographic areas, they come with their own disadvantages. WANs are often subject to higher costs, increased complexity, and potential security challenges due to the vast distance they cover. Moreover, WANs can experience slower data transfer rates compared to LANs or MANs due to the long distances involved.

\end{document}